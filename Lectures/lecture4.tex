\documentclass{beamer}
%
% Choose how your presentation looks.
%
% For more themes, color themes and font themes, see:
% http://deic.uab.es/~iblanes/beamer_gallery/index_by_theme.html
%
\mode<presentation>
{
  \usetheme{Frankfurt}      % or try Darmstadt, Madrid, Warsaw, ...
  \usecolortheme{beaver} % or try albatross, beaver, crane, ...
  \usefonttheme{serif}  % or try serif, structurebold, ...
  \setbeamertemplate{navigation symbols}{}
  \setbeamertemplate{caption}[numbered]
} 

\usepackage[english]{babel}
\usepackage[utf8x]{inputenc}
\usepackage{graphicx}
%\usepackage[backend=biber,style=authoryear]{biblatex}
%\usepackage{minted}
%\usemintedstyle{friendly} % You can also try: xcode, tango, borland, vs, autumn
\usepackage[dvipsnames]{xcolor}
\usepackage{listings}
\definecolor{Gray}{gray}{0.5}
\definecolor{DarkGreen}{rgb}{0,0.4,0}

\lstset{
  language=R,
  basicstyle=\ttfamily\small,
  keywordstyle=\color{Blue},
  commentstyle=\color{Gray},
  stringstyle=\color{DarkGreen},
  numbers=left,
  numberstyle=\tiny\color{Gray},
  breaklines=true,
  showstringspaces=false
}


\title[Your Short Title]{Introduction to probability and statistics}
\author{Master in Cognitive Science 2025-2026}
\institute{Lecture 5 } % Change this later with 4
\date{November, 2025}

\begin{document}

\begin{frame}
  \titlepage
\end{frame}


\begin{frame}{Objectives}
\begin{itemize}
\item Introduction to the basics of statistical inference (\textbf{frequentist} paradigm).
\item Definition of an estimator.
\item Important results: Law of Large Number (LLN) and the Central Limit Theorem (CLT). 
%\item Confidence interval
\end{itemize}
\textbf{Readings}: Estimation Doc\_fr4 file in Ametice. Kass et al Chapters 7.1, 7.2, 8.1
\end{frame}

%%%%%%%%%%%%%%%%%%%%%%%%%%%%%
\section{Basics of statistical inference}
%%%%%%%%%%%%%%%%%%%%%%%%%%%%%
\begin{frame}
\textbf{Definition}: 
\begin{itemize}
\item A population $\mathcal{P}$ is a large (sometimes infinite) set of individuals, objects, statistical units.
\item A sample is a subset $\mathcal{E}$ of a population. It needs to be representative of the population:
\begin{itemize}
\item the sample size $n$ should be large if possible.
\item individuals randomly sampled from the population $\mathcal{P}$.
\end{itemize}
\end{itemize} 

\end{frame}

%%%%%

\begin{frame}

\textbf{Link between probability and statistics}

\begin{itemize}
\item \textit{probability}: $\mathcal{P}$ is known, we compute the probability of observing some events in $\mathcal{E}$.
\item \textit{statistics}: $\mathcal{P}$ is unknown, we observe $\mathcal{E}$ and try to deduce properties of $\mathcal{P}$.
\end{itemize}



\begin{block}{Example}
Consider the following example: the population characteristic is the height of a person. Let $X$ be a quantitative random variable representing a person's height. We can describe it as $$X\sim \mathcal{N}(\mu=175,\sigma^2=15^2)$$.
\end{block}

\end{frame}

%%%%%%

\begin{frame}[fragile]
\begin{lstlisting}
set.seed(123)
Pop = round(rnorm(10000, mean=175, sd = 15))
\end{lstlisting}

We pick randomly a sample $(x_1,\ldots,x_n)$ from the population. 
\begin{lstlisting}
n = 10
samp1 <- sample(Pop)
samp1
\end{lstlisting}
In this  case 
\begin{lstlisting}
## [1] 184 162 175 176 183 146 180 168 189 167
\end{lstlisting}
so if we compute the mean of \textit{samp1}, we get an \textit{estimation} of the mean of the population 
\begin{lstlisting}
mean(samp1)
## [1] 173
\end{lstlisting}
\end{frame}

%%%%%%%%%%%%%%%%%%%%%%%%%%%%%
\section{Descriptive statistics}
%%%%%%%%%%%%%%%%%%%%%%%%%%%%%
\begin{frame}[fragile]
Consider the dataset \textit{bfi} from the package \textit{psychTools}. We are interested on the age of the $2800$ participants.

\begin{lstlisting}
library(psychTools)
data(bfi)
help(bfi)
ages <- bfi$age
\end{lstlisting}

\end{frame}

% %%%%%%%
\begin{frame}[fragile]{Visualization}
\begin{figure}
    \centering
    \includegraphics[width=0.7\textwidth]{figures/histD.png}
    \caption{Histogram and Density of Age}
\end{figure}
\end{frame}

% %%%%%%%
\begin{frame}[fragile]{Visualization of age against other variables}
\begin{figure}
    \centering
    \includegraphics[width=0.85\textwidth]{figures/histAgeGender.png}
    \caption{Comparison of age distributions between male and female}
\end{figure}

\end{frame}

% %%%%%%%
\begin{frame}[fragile]{Measures of location}
We could construct graphical description for our observed sample and/or compute numerical summaries: 
\begin{itemize}
\item \textbf{The empirical mean}: $\overline{x} = \frac{1}{n}\sum\limits_{i=1}^n x_i$.
\item \textbf{Median}: we order our data $x_{(1)} \leq x_{(2)}\leq \cdots\leq x_{(n)}$ then take the value $x_{(\frac{n}{2})}$.
\end{itemize}

\begin{lstlisting}
mean(ages)
## [1] 28.78214

median(ages)
## [1] 26
\end{lstlisting}


\end{frame}
%%%%%%%

\begin{frame}[fragile]{Boxplot}
\begin{figure}
    \centering
    \includegraphics[width=0.7\textwidth]{figures/ageBoxplot.png}
    \caption{Boxplot of Participant Ages}
\end{figure}
\end{frame}

%%%%%%%
\begin{frame}[fragile]{Measure of spread}
We use the variance to measure how far the data values lie from the mean.
\textbf{The empirical variance}: $s^2 = \frac{1}{n} \sum\limits_{i=1}^n (x_i-\overline{x})^2$.

Often, we use the corrected empirical variance $$s_c = \frac{1}{n-1}\sum\limits_{i=1}^n(x_i-\overline{x})^2.$$

\begin{lstlisting}
var(ages)
## [1] 123.8225

sd(ages)
## [1] 11.12755
\end{lstlisting}

\end{frame}
%%%%%%%

\begin{frame}
\begin{figure}
    \centering
    \includegraphics[width=0.85\textwidth]{figures/NormalDist.png}
    \caption{Normal Distributions with Different Variances}
\end{figure}
\end{frame}

%%%%%%%

%%%%%%%%%%%%%%%%%%%%%%%%%%%%%
\section{Estimator}
%%%%%%%%%%%%%%%%%%%%%%%%%%%%%
\begin{frame}
\begin{itemize}
\item We want to model the number of spike train of neuron 1 of \textit{CAL1V} dataset in time interval $5$ and $6$.
Suppose the number is a random variable $X\sim \mathcal{P}(\theta)$ for some $\theta$.

\item In this example there are 20 trials, so we model the number of spikes of each trial $i$ by $X_i\sim\mathcal{P}(\theta)$.

\item The trials are independent, so $X_1,X_2,\cdots, X_{20}$ are \textbf{independent}. Also they have the \textbf{same} distribution; in this case we say that they are \textbf{identically distributed}.

\item We write shortly $X_1,X_2\cdots,X_{20}$ are \textbf{i.i.d} to say that the random variables are independent and identically distributed.
\end{itemize}

\end{frame}

\begin{frame}[fragile]
% picture of the spikes 
\begin{enumerate}
\item From this data we get a realization of $X_1,X_2,\cdots,X_{20}$, which we denote by $x_1,x_2,\cdots,x_{20}$.
\begin{lstlisting}
[1] 51 39 40 70 68 36 44 61 46 61 63 62 49 44 44 51 64 55 47 58
\end{lstlisting}


\item An \textbf{estimator} of the parameter $\theta$ is given by $$T(X_1,X_2,\cdots,X_{20}) = \frac{X_1+X_2+\cdots+X_{20}}{20}.$$


\item An estimation of the parameter $\theta$ is given by $$\hat{\theta} = \frac{x_1 + x_2 + \cdots + x_{20}}{20} = 52.65$$
\end{enumerate}





\end{frame}

\begin{frame}{Estimator}
Let $X_1,\cdots,X_n$ be an i.i.d. It forms a random sample.

The observed sample $x_1,\cdots,x_n$ is viewed as a realization of the random sample.

Suppose the distribution of $X_i$ depends on a parameter $\theta$.

\begin{block}{Definition}
An estimator of a population characteristic $\theta$ is a 'function' $T_n := T(X_1,\cdots,X_n)$ which estimates $\theta$.
\end{block}
We need quality criteria to distinguish estimators. We say that $T_n$ is 
\begin{itemize}
\item unbiased if $E[T_n] = \theta$,
\item efficient if $\text{Var}(T_n)$ is small,
\item consistent if $E[T_n]=\theta$ and $\text{Var}(T_n) \to 0, n\to \infty$.
\item The mean squared error of $T_n$ is defined as $R(T_n,\theta) = E[(T_n-\theta)^2]$.
\end{itemize}
\end{frame}

%%%% 
\begin{frame}{Example}
Consider the estimator $T_{20}$ of the average number of spikes.
\begin{block}{Exercise}
\begin{itemize}
\item Check that the estimator $T_{20}$ is unbiased.
\item Compute the variance of $T_{20}$.
\end{itemize}
\end{block}
\end{frame}

%%%%%%%

\begin{frame}{Bias variance decomposition}
Let $T_n$ be an estimator of a parameter $\theta$, then we have 
\begin{align*}
R(T_n,\theta) &= E\left[(T_n-\theta)^2\right] \\
&= E\left[\left(T_n - E[T_n] + E[T_n] - \theta\right)^2\right]\\
&= E\left[\left(T_n - E[T_n]\right)^2\right] + \left(E[T_n] - \theta\right)\\
&= \text{Var}(T_n) + \text{Bias}^2(T_n).
\end{align*}
We say that we decomposed the mean squared error into the variance of the estimator and the square of its bias.
\end{frame}

%%%
\begin{frame}
\begin{figure}
    \centering
    \includegraphics[width=0.7\textwidth]{figures/BiasVar.png}
    \caption{Bias–Variance Trade-off}
\end{figure}
\end{frame}

%%%%%%%%%%%%%%%%%%%%%%%%%%%%
\section{LLN and CLT}
%%%%%%%%%%%%%%%%%%%%%%%%%%%%
\begin{frame}{Large Law Number}
\begin{block}{Theorem}
Let $X_1,\ldots,X_n$ be i.i.d r.v. (independent identically distributed random variables) with $\mathbb{E}X_i=\mu$ and $Var(X_i)=\sigma^2$. Then, 

$$
\bar{X}_n:= \frac{X_1+X_2+\cdots + X_n}{n}\rightarrow\mu,\ \text{as } n\rightarrow +\infty.
$$
We say the random mean converges to $\mu$ (this holds in a sense that we won't discuss here).
\end{block}
  

\end{frame}

\begin{frame}{Central Limit Theorem}
\begin{block}{Theorem}
Let $X_1, \ldots, X_n$ be i.i.d.\ random variables with $\mathbb{E}[X_i] = \mu$ and $\mathrm{Var}(X_i) = \sigma^2 < \infty$. Then, as $n \to \infty$,
\[
\frac{\sqrt{n}(\bar{X}_n - \mu)}{\sigma} \rightarrow \mathcal{N}(0,1).
\]
That is, the standardized sample mean converges to the standard normal law.
\end{block}
\end{frame}


\end{document}