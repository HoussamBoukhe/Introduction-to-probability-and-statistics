\documentclass[11pt,a4paper]{article}

\textwidth=16cm
\textheight=24.5cm
\oddsidemargin0cm
\topmargin-1cm
\parindent0cm
\parskip0ex
%\linespread{1.2}

\usepackage{natbib}
\bibpunct{(}{)}{;}{a}{,}{,}
\setlength{\bibsep}{0pt}
\newcommand{\biblist}{
\bibliographystyle{apalike}
\bibliography{BiblioRA}
}

\newcommand{\Prob}{\mathrm{P}}
\newcommand{\E}{\mathrm{E}}
\newcommand{\Cov}{\mathrm{Cov}}
\newcommand{\Var}{\mathrm{Var}}
\newcommand{\Cor}{\mathrm{Corr}}


\newcommand{\bds}{\boldsymbol}

\usepackage{times}

\usepackage[english]{babel} %on pourrait ausi utiliser [english] \usepackage{latexSym}
\usepackage{amssymb}
\usepackage{amsmath}
\usepackage{amsfonts}
\usepackage[dvips]{graphicx}
\usepackage{subfigure}
\usepackage{epsfig}
\usepackage{overpic}
\usepackage{rotating}
\usepackage{here}
\usepackage{txfonts}
\usepackage{color}
\usepackage{url}

%\usepackage{geometry} %destroys the textwidth commands
\usepackage{tikz}


\newtheorem{proof}{proof}
%\newcommand{\bpre}{\begin{proof}}
%\newcommand{\epre}{\end{proof}}
\newtheorem{theorem}{Theorem}[section]
\newtheorem{lemma}{Lemma}[section]
\newtheorem{proposition}{Proposition}[section]
\newtheorem{corollary}{Corollary}[section]
\newtheorem{definition}{Definition}[section]
\newtheorem{example}{Example}[section]
\newtheorem{remark}{\bf Remark}[section]
\newtheorem{algorithm}{\bf Algorithm}[section]
\newtheorem{program}{Program}[section]
\newtheorem{note}{\bf Note}
\newenvironment{example1}[1][ ]{\begin{example}[#1]\em}{\qed\end{example}}
\newenvironment{remark1}[1][ ]{\begin{remark}[#1]\em}{\end{remark}}

\newcommand{\bit}{\begin{itemize}}
\newcommand{\eit}{\end{itemize}}
\newcommand{\eps}{\varepsilon}
\newcommand{\eqd}{\stackrel{d}{=}}
\newcommand{\law}{\mathcal{L}}
\newcommand{\rmd}{\mathrm{d}}
\newcommand{\e}{\mathrm{e}}
\newcommand{\floor}[1]{\lfloor #1 \rfloor}
\newcommand{\norm}[1]{\left\| #1 \right\|}
\newcommand{\1}{\boldsymbol{1}}

\newcommand{\DD}{\mathbb{D}}
\newcommand{\PP}{\mathbb{P}}
\newcommand{\RR}{\mathbb{R}}
\newcommand{\XX}{\mathbb{X}}
\newcommand{\mbN}{\mathbb{N}}
\newcommand{\mbZ}{\mathbb{Z}}

\newcommand{\dx}{\mathrm{d}x}
\newcommand{\argmin}{\mathop{\mathrm{arg\,min}}}

\newcommand{\dto}{\rightsquigarrow}
\newcommand{\pto}{\stackrel{\mathrm{P}}{\to}}

\newcommand{\bX}{\overline{X}}
\newcommand{\bXX}{\overline{X^2}}
\newcommand{\bY}{\overline{X}}
\newcommand{\bYY}{\overline{X^2}}
\newcommand{\bXY}{\overline{XY}}


\def\Prodi{\mathop{{\lower 3pt\hbox{\epsfxsize=15pt\epsfbox{pi.ps}}}}}
\def\prodi{\mathop{{\lower 1pt\hbox{\epsfxsize=8pt\epsfbox{pi.ps}}}}}
\input epsf.sty
\newcommand{\CAL}[1]{\mathcal{#1}}
\newcommand{\si}{\mbox{$\sigma$}}
\newcommand{\epsi}{\mbox{$\varepsilon$}}
\newcommand{\sisq}{\mbox{$\sigma^2$}}
\newcommand{\al}{\mbox{$\alpha$}}
\newcommand{\be}{\mbox{$\beta$}}
\newcommand{\vfi}{\mbox{$\varphi$}}
\newcommand{\beq}{\begin{equation}}
\newcommand{\eeq}{\end{equation}}
\newcommand{\p}{\mbox{$\mathcal{P}$}}
\newcommand{\Q}{\mbox{$\mathcal{Q}$}}
\newcommand{\bea}{\begin{eqnarray}}
\newcommand{\eea}{\end{eqnarray}}
\newcommand{\beas}{\begin{eqnarray*}}
\newcommand{\eeas}{\end{eqnarray*}}
\newcommand{\ind}{1\hspace{-2.5mm}{1}}
\newcommand{\BB}[1]{\mathbb{#1}}
\newcommand{\noi}{\noindent}

\def\tcr{\textcolor{red}}

\usepackage[printsolution=false]{exercises}

\begin{document}

\begin{center}
\huge Probability and statistics.   \\
\large Master in Cognitive Science. Academic year 2025-2026.\\
\large Example sheet 2.
\end{center}


%#################################################################################

\begin{exercise}
A pharmaceutical company decided to make savings on mailing advertisement for clients. Therefore, they decided to randomly stamp "urgent" 3 letters over 5 and the others as "normal".

\begin{enumerate}
    \item  Four letters are sent to a medical center where four doctors are working. What is the probability of:
\begin{itemize}
    \item A : ”At least one of them get the letter with urgent stamp”
    \item B : ”exactly 2 doctors get the letter with urgent stamp
\end{itemize}
\item Let $X$ be the random variable ”number of letters with urgent stamp among 10 letters.
\begin{itemize}
    \item What is the probability distribution of X?
    \item What are its expectation and variance?
\end{itemize}
\end{enumerate}
\end{exercise}

\begin{solution}
\begin{enumerate}
\item 
\begin{itemize}
\item $A=$"At least one letter with urgent stamp". Define $B_i=$"letter i with normal stamp". The probability that all letters have normal stamp is $P(B)=P(B_1\cap B_2\cap B_3\cap B_4) =P(B_1)P(B_2)P(B_3)P(B_4)=\left(\frac{2}{5}\right)^4$. Hence, $P(A)=1-P(B)=0.975.$

\item Exactly two with urgent stamps. Pick two among 4 with probability of success $\frac{3}{5}$.

$$
\binom{4}{2}\left(\frac{2}{5}\right)^2\left(\frac{3}{5}\right)^2=\frac{4!}{2!(4-2)!}\left(\frac{3}{5}\right)^2=0.3456.
$$
\end{itemize}


\item $X \sim Binom(n=10,\theta=3/5)$. $\mathbb{E}(X)=n\theta=6$, $Var(X)=n\theta(1-\theta)=2.4$.
 
\end{enumerate}
\end{solution}

%#################################################################################
\begin{exercise}
We denote as $X$ the random variable modeling the number of purchases on the Amazon book website. We assume it is Poisson distributed with parameter $\lambda$. We know that the average number of purchases per second is 10.

\begin{enumerate}
    \item Compute the probability of there being $9$, $10$ or $11$ purchases per second.
    \item Compute the probability of there being fewer than 2 purchases per second.
    \item Suppose that instead of knowing that the average number of purchases per second is $10$, you are told that $P(X = 0) = 0.082$. Find $\lambda$ and $Var(X)$.
\end{enumerate}

\end{exercise}

\begin{solution}
\noindent Let $X=\text{"number of purchase per day"}$. $X\sim Poi(\lambda),\ \lambda>0$.

\begin{itemize}
    \item $\mathbb{E}(X)=\lambda=10$. Hence, $P(X=9)=0.125;P(X=10)=0.125;P(X=11)=0.114.$ \textit{In R: use 'dpois()'.}
    
    \item $P(X\leq 2) = F_x(2)=\sum_{k\in\left\{0,1,2\right\}}P(X=k)=2.8.10^{-3}.$ \textit{In R: use 'ppois()'.}
    
    \item $P(X=0)=0.082$, from which we obtain $\lambda = 2.5$. For a Poisson distributed r.v, we know that $\mathbb{E}(X)=\mathbb{V}ar(X)=\lambda$.
    
\end{itemize}
\end{solution}

%#################################################################################

\begin{exercise}
Suppose that $X$ has the probability density function $f(x)= c(1-x^2)$ for $0\leq x\leq 1$ and $f(x)=0$ otherwise.

\begin{enumerate}
\item Find c.
\item Compute the cumulative density function (cdf).
\item Compute $P(0.1 \leq  X \leq 0.9)$.
\item Find $x_{0.95}$ such that $P(X\leq x_{0.95} )= 0.95$.
\end{enumerate}
\end{exercise}

\begin{solution}
\begin{itemize}
    \item $c=3/2$.
    
    \item $F(x)=P(X\leq x)=-\frac{x}{2}\left[x^2-3\right]$.
    
    \item $P(0.1 \leq  X \leq 0.9)= 0.836$.
    
    \item $q$ is solution of the equation $q_{0.95}^3-3q_{0.95}+1.9=0$, only one of those solution belongs to $[0,1]$; $q_{0.95}=0.8114$.
    
\noindent     You can use R.
\begin{verbatim}
    library(RConics)
    cubic(p=c(1,0,-3, 1.9)
\end{verbatim}    
    
\end{itemize}
\end{solution}




%################

\begin{exercise}
Suppose that the lifetime of light bulb follows an exponential distribution with parameter $\lambda = 0.1$.
\begin{enumerate}
\item What is the probability that the lifetime is less than 10.
\item What is the probability that the lifetime is between 5 and 15.
\item Find $t$ such that the probability that the lifetime is greater than $t$ is 0.01.
\item (*) In an experiment you observe the following lifetimes $(1.16, 3.51, 7.90, 4.16, 2.07, 1.96)$ in thousands of hours. Using \texttt{Rjags}, obtain the distribution (posterior) of the parameter $\lambda$. \textit{Use a Gamma prior distribution}.
\end{enumerate}
\end{exercise}

\begin{solution}
Let $X=\text{"bulb lifetime}"$. $X\sim Exp(\lambda),\ \lambda=0.1$.

\begin{itemize}
    \item  The cdf of X is given by 
$$
F(x)=\int_0^x\lambda e^{-\lambda u}du=1-e^{-\lambda x}.
$$

\item $P(X<10)=F(10)=1-e^{-10*0.1}=0.6321$.

\item $P(R\leq X\leq 10)=F(10)-F(5)=0.6321 - 0.3935 = 0.2386$

\item $P(X>t)=0.01$.
$$
P(X\leq t)=0.99\\
$$
i.e., $t=\frac{-\log(0.01)}{0.1}=46.$

\end{itemize}

\end{solution}



\end{document}

%#################################################################################


\begin{exercise}
The cdf of the Rayleigh probability model is given by
$$
F(x) = 1-\exp\left\{-\frac{x^2}{2\sigma^2}\right\}
$$
for $x>0$ and $\sigma>0$.
\bit
\item Find the pdf.
\item Find the $0.05$ and $0.975$ percentiles.
\item Compute $P(0.25<X<0.75)$
\eit

\end{exercise}

\begin{solution}
.
\end{solution}
